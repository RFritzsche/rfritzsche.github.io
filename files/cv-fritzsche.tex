%%%%%%%%%%%%%%%%%%%%%%%%%%%%%%%%%%%%%%%%%%%%%%%%%%%%%%%%%%%%%%%%
%%% Compilation: 
%%% PDFLaTeX BibTeX PDFLaTeX PDFLaTeX
%%% Antonio Machicao y Priemer
%%%%%%%%%%%%%%%%%%%%%%%%%%%%%%%%%%%%%%%%%%%%%%%%%%%%%%%%%%%%%%%%


\documentclass[11pt]{article}

%%%%%%%%%%%%%%%%%%%%%%%
%%% Packages:
%%%%%%%%%%%%%%%%%%%%%%%

\usepackage[utf8]{inputenc}
%\usepackage{xcolor}		%Farbelemente
%\usepackage{natbib}		%Bibliographie
\usepackage[ngerman,english]{babel}
\usepackage{setspace}	%Zeilenabstand
	\onehalfspacing
\usepackage[sc]{titlesec} %Small-Caps Sections

\usepackage{tabularx}
\usepackage{ragged2e}
\usepackage{multirow}  %Mehrere Zeilen in einer Tabelle
%\usepackage{tipa}		%IPA-Zeichen
\usepackage{amsmath}
\usepackage{amsfonts}
\usepackage{amssymb}
\usepackage{MnSymbol} %Mathematische Klammern und Symbole (Inkompatibel mit ams-Packages!) 
\usepackage{xspace}		%Leerzeichen
\usepackage[dvipdfm
%,showframe
]{geometry}	%Ränder
\usepackage{array}
\usepackage[bottom]{footmisc}	%FN unten!
\usepackage{listings}	%Besondere Listen
%\geometry{left=3cm,right=3cm,vmargin=3cm}
\usepackage{marginnote}	%Notizen
\usepackage{upgreek}

% Letzte Pakete 
\usepackage[hidelinks]{hyperref}	%URLs


%%%%%%%%%%%%%%%%%%%%%%%
%%% New Commands:
%%%%%%%%%%%%%%%%%%%%%%%

%% Quotation marks (German):
\newcommand{\gqq}[1]{\glqq{}#1\grqq{}}
\newcommand{\gq}[1]{\glq{}#1\grq{}}

%% German Abbreviations: \,	
\newcommand{\idR}{\mbox{i.\,d.\,R.}\xspace}
\newcommand{\su}{\mbox{s.\,u.}\xspace}
\newcommand{\ua}{\mbox{u.\,a.}\xspace}
\newcommand{\zB}{\mbox{z.\,B.}\xspace}
%\dh --> d.\,h.

%% Rightarrow
\def\ra{\ensuremath\rightarrow}
\def\ras{\ensuremath\rightarrow\ }

%% Notes
\renewcommand{\marginfont}{\singlespacing}
\renewcommand{\marginfont}{\footnotesize}
\renewcommand{\marginfont}{\color{black}}

\newcommand{\myp}[1]{%
	\marginnote{%
		\begin{spacing}{1}
			\vspace{-\baselineskip}%
			\color{red}\footnotesize#1
		\end{spacing}
	}
}

%%%%%%%%%%%%%%%%%%%%%%%
%%% End Preamble
%%%%%%%%%%%%%%%%%%%%%%%

\begin{document}

\pagestyle{empty}

%%%%%%%%%%%%%%%%%%%%%%%
%% Title
\noindent\begin{minipage}{.48\textwidth}

\begin{flushleft}
\huge{\textsc{Rosa Fritzsche}}
\vspace{.5\baselineskip}

\Large{\textsc{Curriculum Vitae}}\\
\vspace{\baselineskip}
\small{Updated: \today}
\end{flushleft}
\end{minipage}\hfill%
\begin{minipage}{.48\textwidth}
\begin{flushright}
	Universität Leipzig\\
	Institut für Linguistik (IGRA)\\
	Beethovenstr. 15, 04107 Leipzig\\
%	+49~341~9739786 \\
	\href{mailto:r.fritzsche@uni-leipzig.de}{r.fritzsche@uni-leipzig.de} \\
	\href{http://rosafritzsche.de/}{http://rosafritzsche.de/}\\
\end{flushright}
\end{minipage}
%\vspace{1.5\baselineskip}

\noindent\rule{\textwidth}{1pt}


%%%%%%%%%%%%%%%%%%%%%%%
%\section*{Contact Information}

%\begin{flushleft}
%  Edlichstr. 20, 04315 Leipzig \\
%  Telephone: +49~176~61420251 \\
%  Email: \href{mailto:robert@robertfritzsche.de}{robert@robertfritzsche.de} \\
%  Homepage: \url{https://www.linguistik.hu-berlin.de/staff/amyp}\\
%\end{flushleft}


%%%%%%%%%%%%%%%%%%%%%%%

%%%%%%%%%%%%%%%%%%%%%%%
%\section*{Persönliche Daten}
%\begin{flushleft}
%  %Gender: Male \\
%  Geburtsdatum: 16.05.1991 \\
%  Geburtsort: Chemnitz \\
%  Staatsangehörigkeit: deutsch \\
%  Familienstand: ledig \\
%\end{flushleft}

%\pagebreak


%%%%%%%%%%%%%%%%%%%%%%%
\section*{Education}



\begin{flushleft}
	\begin{tabularx}{\textwidth}{@{}p{0.2\textwidth}>{\RaggedRight\arraybackslash}p{.8\textwidth}@{}}
		04/2019 - current & \textbf{PhD in Linguistics}, \textit{Universität Leipzig}\\
		& Research Training Program `Interaction of Grammatical Building Blocks' (IGRA) \\
		& Thesis: Anti-local agreement \\
		& Reviewers: Gereon Müller, Ian Roberts (University of Cambridge) \\
		& \\
		10/2015 - 09/2018 & \textbf{MA Linguistics}, \textit{Humboldt-Universität zu Berlin} \\
		& MA Thesis: \textit{Morpho-syntax of verbal \textit{-s} in North Eastern English} \\
		& Advisors: Artemis Alexiadou, Hubert Truckenbrodt\\
		& Final Grade: 1,3\\
		& \\
		10/2011 - 09/2015 & \textbf{BA English studies}, \textit{Universität Leipzig} \\
	\end{tabularx}
\end{flushleft}

%%%%%%%%%%%%%%%%%%%%%%%
\section*{Employment}

\begin{flushleft}
	\begin{tabularx}{\textwidth}{@{}p{0.2\textwidth}>{\RaggedRight\arraybackslash}p{.8\textwidth}@{}}
		04/2019 - 03/2023 & \textbf{Researcher (wissenschaftliche Mitarbeiterin)}, \textit{Universität Leipzig}, Research Training Program `Interaction of Grammatical Building Blocks' (IGRA) \\
		& \\
		07/2016 - 09/2018 & \textbf{Research assistant (studentische Hilfskraft)}, \textit{Humboldt Universität zu Berlin}, Department of German Studies and Linguistics \\
	\end{tabularx}
\end{flushleft}



%%%%%%%%%%%%%%%%%%%%%%%

\section*{Publications}
\subsubsection*{Peer-reviewed}
\begin{flushleft}
	\begin{tabularx}{\textwidth}{@{}p{0.2\textwidth}>{\RaggedRight\arraybackslash}p{.8\textwidth}@{}}
		2023 & \textit{Ordering discontinuous $\upvarphi$-feature Agree: Verbal -s in North Eastern English}. Journal of Comparative Germanic Linguistics 26: 7. \\
	\end{tabularx}
\end{flushleft}


\subsubsection*{Working Papers}
\begin{flushleft}
	\begin{tabularx}{\textwidth}{@{}p{0.2\textwidth}>{\RaggedRight\arraybackslash}p{.8\textwidth}@{}}
	2023 & \textit{Anti-Local Agree and Cyclicity}. In: M. Privizentseva, F. Andermann and G. Müller, eds, \textit{Cyclicity}. Vol. 95 of Linguistische Arbeits Berichte, Institut für Linguistik, Universität Leipzig. \\
\end{tabularx}
\end{flushleft}
%%%%%%%%%%%%%%%%%%%%%%%

\section*{Presentations}
\subsubsection*{Peer-reviewed}

\begin{flushleft}
	\begin{tabularx}{\textwidth}{@{}p{0.2\textwidth}>{\RaggedRight\arraybackslash}p{.8\textwidth}@{}}
		2021 & \textit{Mutual counterfeeding in Bari as two separate counterfeeding interactions}. Talk presented at the 18th Old World Conference on Phonology (OCP18), UIB Eivissa.\\
		& \\
		2021 &  \textit{Mutual counterfeeding in Bari as two separate
		 counterfeeding interactions}. Talk presented at 29th Conference of the Student Organization of Linguistics in Europe, Leiden University Centre for Linguistics. \\
		 & \\
		 2020 & \textit{Mutual counterfeeding in Bari as two separate counterfeeding interactions}. Poster presented at the Annual Meeting on Phonology (AMP) 2020, UC Santa Cruz.\\
	\end{tabularx}
\end{flushleft}

\section*{Teaching}
\subsubsection*{Instructor}
\begin{flushleft}
	\begin{tabularx}{\textwidth}{@{}p{0.2\textwidth}>{\RaggedRight\arraybackslash}p{.8\textwidth}@{}}
		Summer 2022 & \textbf{Übung Morphologie}, BA Linguistics, Universität Leipzig \\
		& \\
		Summer 2019 & \textbf{Varieties of English}, BA English Studies, Humboldt Universität zu Berlin \\
	\end{tabularx}
\end{flushleft}
\subsubsection*{Teaching Assistant}	
\begin{flushleft}
	\begin{tabularx}{\textwidth}{@{}p{0.2\textwidth}>{\RaggedRight\arraybackslash}p{.8\textwidth}@{}}
		Summer 2017 & \textbf{Tutorial Germanic Syntax}, BA Linguistics, Humboldt Universität zu Berlin \\
	\end{tabularx}
\end{flushleft}


\section*{Additional Training}
\begin{flushleft}
	\begin{tabularx}{\textwidth}{@{}p{0.2\textwidth}>{\RaggedRight\arraybackslash}p{.8\textwidth}@{}}
	2019	& \textbf{Eastern Generative Grammar Summer School}, University of Wrocław, 29 July -- 9 August 2019 \\
	\end{tabularx}
\end{flushleft}

%%%%%%%%%%%%%%%%%%%%%%%
\section*{Scholarships}
\begin{flushleft}
	\begin{tabularx}{\textwidth}{@{}p{0.2\textwidth}>{\RaggedRight\arraybackslash}p{.8\textwidth}@{}}
		01/2014 - 03/2018 & Basic grant from \textbf{Friedrich-Ebert-Stiftung} \\
	\end{tabularx}
\end{flushleft}


%%%%%%%%%%%%%%%%%%%%%%%
\section*{Skills}
\begin{flushleft}
	\begin{tabularx}{\textwidth}{@{}p{0.2\textwidth}>{\RaggedRight\arraybackslash}p{.8\textwidth}@{}}
	Languages	& German (native), English (fluent), French (basic), Modern Hebrew (basic), Latin (Latinum certificate) \\
	& \\
	IT & LaTeX (expert), ANNIS Corpus Tools (advanced), Python (basic), R (basic), Microsoft Office\\
	\end{tabularx}
\end{flushleft}
	

%%\section*{Referees}
%\begin{flushleft}
%	\begin{tabularx}{\textwidth}{@{}>{\RaggedRight\arraybackslash}p{.5\textwidth}>{\RaggedRight\arraybackslash}p{.5\textwidth}@{}}
%	Gereon Müller	 & Ian Roberts  \\
%	Universität Leipzig  & University of Cambridge  \\
%	 gereon.mueller@uni-leipzig.de & igr20@cam.ac.uk \\
%	 & \\
%	Artemis Alexiadou & \\
%	Leibniz-Zentrum Allgemeine  & \\
%	Sprachwissenschaft & \\
%	artemis@leibniz-zas.de & \\
%	\end{tabularx}
%\end{flushleft}



\end{document}

%%%%%%%%%%%%%%%%%%%%%%%
%Tabellen
\begin{table}[htbp] \centering%
\begin{tabular}{lll}\hline\hline
1 & 2 & 3 \\ \hline
1 & \multicolumn{2}{c}{2} \\
\hline
\end{tabular}
\caption{Titel\label{Tabelle: Label}}
\end{table}






